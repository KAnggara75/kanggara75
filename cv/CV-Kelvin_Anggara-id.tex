% docx2tex 1.7.3 --- ``Just out of this Word.'' 
% 
% docx2tex is Open Source and  
% you can download it on GitHub: 
% https://github.com/transpect/docx2tex 
%  
\documentclass{scrbook} 
\usepackage[T1]{fontenc} 
\usepackage[utf8]{inputenc} 
\usepackage{hyperref} 
\usepackage{multirow} 
\usepackage{tabularx} 
\usepackage{color} 
\usepackage{textcomp} 
\usepackage{tipa}
\usepackage{amsmath} 
\usepackage{amssymb} 
\usepackage{amsfonts} 
\usepackage{amsxtra} 
\usepackage{wasysym} 
\usepackage{isomath} 
\usepackage{mathtools} 
\usepackage{txfonts} 
\usepackage{upgreek} 
\usepackage{enumerate} 
\usepackage{tensor} 
\usepackage{pifont} 
\usepackage{ulem} 
\usepackage{xfrac} 
\usepackage{soul}
\usepackage{arydshln} 
\usepackage[english,bahasa]{babel}

\begin{document}

\begin{table}
\begin{tabularx}{\textwidth}{
p{\dimexpr 1\linewidth-2\tabcolsep}}
\centering\arraybackslash{}\textbf{KELVIN ANGGARA} \\
\end{tabularx}
\end{table}

\begin{center}Garuda Sakti Km.1, Pekanbaru Riau {\textbar} \href{https://wa.me/6282284705204}{+62 822 8470 5204} {\textbar} kelvin.kanggara@gmail.com
\end{center}


\begin{center}\href{https://linkedin.com/in/kanggara75}{linkedin.com/in/kanggara75} {\textbar} \href{https://github.com/kanggara75}{github.com/kanggara75} \foreignlanguage{english}{{\textbar}} \href{https://kanggara.me/}{kanggara.me}
\end{center}

\uppercase{\textbf{\uline{Tentang saya}}} 

Saya Kelvin Anggara, saya memiliki latar belakang Pendidikan Teknik Elektro konsentrasi Teknik Komputer di Universitas Islam Negeri Sultan Syarif Kasim Riau. Saya juga memiliki pengalaman sebagai teknisi komputer \foreignlanguage{english}{dan} jaringan di Perusahaan Penyedia Jaringan Internet Swasta. Saya memiliki ketertarikan kuat untuk berkarier dibidang komputer dan teknologi.

\uppercase{\textbf{\uline{Pendidikan}}} 

\textbf{UIN SULTAN SYARIF KASIM RIAU} Pekanbaru, Riau Indonesia

\textbf{Sarjana Teknik \textendash{} Teknik Elektro} \textendash{} IPK: 3.52 dari 4.00 2016-2022

Judul Tugas Akhir/Skripsi:

\textbf{\textit{Smart Early Warning System}} \textbf{untuk keamanan Sepeda Motor Berbasis Prosesor XTensa LX6}
\begin{itemize}
\item Asisten Praktikum Fisika 2020
\item Asisten Praktikum Komputer dan Jaringan 2019-2020
\item \textit{Project} \textit{Mini} (Sistem Keamanan pada Sepeda Motor terintegrasi Aplikasi Android) 2019
\item Asisten Lab. Sistem Tertanam/\textit{Embedded} \textit{Systems} 2019
\end{itemize}
\textbf{SMK NUSANTARA ROKAN HILIR} Rokan Hilir, Riau Indonesia 

Teknik Komputer dan Jaringan 2014-2016
\begin{itemize}
\item Magang di: CV. Hafizah Network 2015
\item Ujian Kompetensi Oleh: PT. Telekomunikasi Indonesia, Tbk (Bagan Batu) 2016
\end{itemize}
\uppercase{\textbf{\uline{Pelatihan \& sertifikasi}}} 

\uppercase{\textbf{Coursera}}
\begin{itemize}
\item \href{https://coursera.org/verify/KHQWR9GBV8JN}{Dasar - Dasar Dukungan Teknis} 2022
\item \href{https://coursera.org/verify/D936FCCK8JXC}{Seluk Beluk Jaringan Komputer} 2022
\end{itemize}
\uppercase{\textbf{BPPTIK}}
\begin{itemize}
\item Junior Web Developer \uppercase{2021}
\item Junior Office Operator 2021
\end{itemize}
\uppercase{\textbf{Dicoding Academy}} 
\begin{itemize}
\item Belajar Membuat Aplikasi Flutter untuk Pemula 2021
\item Belajar Dasar Git dengan Github  2021
\item Belajar Prinsip Pemrograman SOLID 2021
\item Memulai pemrograman dengan DART \uppercase{2020}
\item Belajar membuat aplikasi Android untuk pemula \uppercase{2020}
\end{itemize}
\uppercase{\textbf{\uline{Pengalaman Kerja}}} 

\uppercase{\textbf{Cv. Hafizh Network}} Rokan Hilir, Riau Indonesia

{\textbullet}~Teknisi Komputer dan Jaringan \uppercase{2016}

Bertanggung jawab melakukan instalasi, konfigurasi, perawatan dan perbaikan komputer, \textit{printer}, termasuk perangkat-perangkat jaringan yang terhubung. Melakukan \textit{backup} data secara berkala untuk menghindari kehilangan data. 

\uppercase{\textbf{\uline{Keahlian \& kompetensi}}} 
\begin{itemize}
\item Versioning Control System (Git)
\item DBMS (MySQL, Firebase)
\item Intermediate Computer \& Networking Skill 
\item Microcontroller Programing (Arduino, ESP8266, ESP32)
\item Mobile Apps Developer (Flutter)
\item Web Developer (PHP, HTML, CSS, JavaScript)
\item Web Framework (CodeIgniter, Laravel, Bootstrap CSS, Tailwind CSS)
\end{itemize}
\uppercase{\textbf{\uline{Pengalaman kerja sukarela}}} 

	\uppercase{\textbf{SNTIKI}}	Pekanbaru, Riau Indonesia

{\textbullet}~Panitia SNTIKI ke-12 2020

Sebagai Editor\uppercase{: }Melakukan pengecekan terhadap kesalahan pengetikan/\textit{typo} dan menyesuaikan sistem penulisan artikel sesuai dengan ketentuan yang telah di berikan.

\uppercase{\textbf{\uline{Publikasi}}} 

\textbf{Buku} Indonesia

{\textbullet}~Teknik Membangun Sistem Pengaman Kendaraan Bermotor Terintegrasi Android 2020

ISBN: \href{https://isbn.perpusnas.go.id/Account/SearchBuku?searchTxt=978-602-476-764-8&searchCat=ISBN}{978-602-476-764-8}

Publisher: CV. Kekata Group

\textbf{Artikel Jurnal} Indonesia
\begin{description}[JURNAL]
\item[]{\textbullet}~\textit{Smart Early Warning} System Untuk Keamanan Sepeda Motor Berbasis Prosesor  2021

XTensa LX6 (Vol. 10, No. 2)
\item[JURNAL]: Jurnal Sains dan Teknologi (JST)
\begin{description}[P-ISSN]
\item[P-ISSN]: 2303-3142
\item[E-ISSN]: 2548-8570
\end{description}
		DOI	: \href{https://doi.org/10.23887/jstundiksha.v10i2.33425}{10.23887/jstundiksha.v10i2.33425}
\item[URL]: \href{https://ejournal.undiksha.ac.id/index.php/JST/article/view/33425}{\textit{ejournal}.undiksha.ac.id/\textit{index}.php/JST/\textit{article}/\textit{view}/33425} 
\end{description}

\end{document}
