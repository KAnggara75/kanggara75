\documentclass[11pt,a4paper]{article}

\usepackage[left=2.5cm, right=2cm, top=1.5cm, bottom=1.5cm]{geometry} % set document margins
\usepackage[parfill]{parskip} % Remove paragraph indentation


\usepackage{hyperref} % Hyperlink package
\hypersetup{colorlinks=true, linkcolor=blue, urlcolor=black} % Hyperlink Config

\usepackage[bahasa]{babel}
\usepackage{graphicx} %LaTeX package to import graphics
\graphicspath{{img/}} %configuring the graphicx package

\usepackage{xcolor}


\usepackage{charter} % FONT OPTIONS

% ------------ Utilities ------------

\newcommand{\email}[1]{\href{mailto:#1}{#1}} % Ling to email
\newcommand{\profilelink}[1]{\href{https://#1}{#1}} % to website
% -----------------------------------

\begin{document}

\scriptsize\color{gray}Last updated on \today

\color{black}

% Header Name
\begingroup
\hfill{\MakeUppercase{\Huge\bfseries Kelvin Anggara}}\hfill
\break
\vspace{-1em}
\hrule
\vspace{0.2em}
\hrule
\hrule
\vspace{-0.5em}
\hfill\href{mailto:kelvin.kanggara@gmail.com}{kelvin.kanggara@gmail.com}\hfill
\break
\endgroup

First document. This is a simple example, with no
extra parameters or packages included.
We have now added a title, author and date to our first \LaTeX{} document!
% This line here is a comment. It will not be typeset in the document.

\href{mailto:kelvin.kanggara@gmail.com}{kelvin.kanggara@gmail.com}

Some of the \textbf{greatest}
discoveries in \underline{science}
were made by \textbf{\textit{accident}}.

Some of the greatest \emph{discoveries} in science
were made by accident.

\textit{Some of the greatest \emph{discoveries}
    in science were made by accident.}

\textbf{Some of the greatest \emph{discoveries}
    in science were made by accident.}

% The \includegraphcs command is 
% provided (implemented) by the 
% graphicx package

\begin{figure}[h]
    \centering
    \includegraphics[width=0.75\textwidth]{test.png}
    \caption{A nice plot.}
    \label{fig:test}
\end{figure}

\begin{itemize}
    \item The individual entries are indicated with a black dot, a so-called bullet.
    \item The text in the entries may be of any length.
\end{itemize}

\begin{enumerate}
    \item This is the first entry in our list.
    \item The list numbers increase with each entry we add.
\end{enumerate}


In physics, the mass-energy equivalence is stated
by the equation $E=mc^2$, discovered in 1905 by Albert Einstein.


\begin{math}
    E=mc^2
\end{math} is typeset in a paragraph using inline math mode---as is $E=mc^2$, and so too is \(E=mc^2\).

The mass-energy equivalence is described by the famous equation
\[ E=mc^2 \] discovered in 1905 by Albert Einstein.

In natural units ($c = 1$), the formula expresses the identity
\begin{equation}
    E=m
\end{equation}


Subscripts in math mode are written as $a_b$ and superscripts are written as $a^b$. These can be combined and nested to write expressions such as

\[ T^{i_1 i_2 \dots i_p}_{j_1 j_2 \dots j_q} = T(x^{i_1},\dots,x^{i_p},e_{j_1},\dots,e_{j_q}) \]

We write integrals using $\int$ and fractions using $\frac{a}{b}$. Limits are placed on integrals using superscripts and subscripts:

\[ \int_0^1 \frac{dx}{e^x} =  \frac{e-1}{e} \]

Lower case Greek letters are written as $\omega$ $\delta$ etc. while upper case Greek letters are written as $\Omega$ $\Delta$.

Mathematical operators are prefixed with a backslash as $\sin(\beta)$, $\cos(\alpha)$, $\log(x)$ etc.

\section{First example}

The well-known Pythagorean theorem \(x^2 + y^2 = z^2\) was proved to be invalid for other exponents, meaning the next equation has no integer solutions for \(n>2\):

\[ x^n + y^n = z^n \]

\section{Second example}

This is a simple math expression \(\sqrt{x^2+1}\) inside text.
And this is also the same:
\begin{math}
    \sqrt{x^2+1}
\end{math}
but by using another command.

This is a simple math expression without numbering
\[\sqrt{x^2+1}\]
separated from text.

This is also the same:
\begin{displaymath}
    \sqrt{x^2+1}
\end{displaymath}

\ldots and this:



\begin{abstract}
    This is a simple paragraph at the beginning of the
    document. A brief introduction about the main subject.
\end{abstract}

After our abstract we can begin the first paragraph, then press ``enter'' twice to start the second one.

This line will start a second paragraph.

I will start the third paragraph and then add \\ a manual line break which causes this text to start on a new line but remains part of the same paragraph. Alternatively, I can use the \verb|\newline|\newline command to start a new line, which is also part of the same paragraph.

\end{document}