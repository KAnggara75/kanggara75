\documentclass[a4paper]{article}
\begin{document}
Kelvin Anggara

De stelling van Pythagoras zegt dat a in het kwadraat plus b in het kwadraat gelijk is aan c in het kwadraat.
In een formule: a2+b2=c2. Of in een display:

a2+b2=c2.

Op de middelbare school beweest de jonge Gauss reeds dat dat de som van getallen 1, 2, 3, ..., 100 gelijk is aan 1/2 x 100 x 101. Meer in het algemeen zag hij al in dat:

sum k=1 n = 1/2 n(n+1).

Smaken verschillen, maar de formule e i pi = -1 wordt een van de mooiste formules ter wereld gevonden, omdat hier alle belangrijke symbolen uit de wiskunde in voorkomen.

Typeset zelf een formule die je mooi vindt. Heb je geen inspiratie, neem dan iets over uit Stochastiek 1. Mijn favoriete formule (of vergelijking of definitie) is:

mijn-favoriete-formule
\end{document}