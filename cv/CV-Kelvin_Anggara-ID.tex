\documentclass[11pt,a4paper]{article}

\usepackage[T1]{fontenc} % Encoding for pipe symbol
\usepackage[parfill]{parskip} % Remove paragraph indentation
\usepackage[left=2.5cm, right=1.5cm, top=1.5cm, bottom=1.5cm]{geometry} % set document margins

\usepackage{xcolor} % Font Color with string
\usepackage{xstring} % for sub string
\usepackage{fancyhdr} % Add header and Footer
\usepackage{ragged2e} % Alignment package
\usepackage{hyperref} % Hyperlink package
\usepackage{fontawesome5}
\usepackage{blindtext}
\usepackage{multicol}

% URL link config
\urlstyle{same}
\hypersetup{colorlinks=true, linkcolor=black, urlcolor=black} % Hyperlink Config

% ------------ Utilities ------------
\newcommand\tab[1][1cm]{\hspace*{#1}}
\newcommand\nv[1][-0.5em]{\vspace{#1}}
\newcommand{\email}[1]{\href{mailto:#1}{#1}} % Ling to email
\newcommand{\github}[1]{\href{https://github.com/#1}{#1}} % to Github
\newcommand{\linkedin}[1]{\href{https://linkedin.com/in/#1}{#1}} % to LinkedIn
\newcommand{\web}[1]{\href{https://#1}{#1}} % to Home Page

\newcommand{\wa}[1]{\href{https://wa.me/#1}{
    \;\StrMid{#1}{1}{2}
    \;\StrMid{#1}{3}{5}
    \;\StrMid{#1}{6}{9}
    \;\StrMid{#1}{10}{13}
  }
} % to WhatsApp
% -----------------------------------

% Header and footer Setup
\pagestyle{fancy}
\fancyhf{} % clear all header and footer fields
\renewcommand{\headrulewidth}{0pt}
\renewcommand{\footrulewidth}{0pt}
\fancyfoot{} % clear all header fields
\fancyfoot[RO]{\tiny\color{gray}\textrm{Last updated on \today}}

% -------------------- CUSTOM COMMANDS --------------------
\newcommand{\pipe}{\,\,|\,\,}

\newcommand{\resumeHeading}[1]{
\hfill{\MakeUppercase{\huge\scshape\bfseries #1}}\hfill
\break
\vspace{-1em}
\hrule
}

\newcommand{\resumeInfo}[1]{
\vspace{-0.2em}
\hfill
\faIcon{home}\,\,{Tanggerang Selatan, Banten}\pipe
\faIcon{whatsapp}\,\,\wa{628992199373}\pipe
\faIcon{envelope}\,\,\email{kelvin.kanggara@gmail.com}
\hfill
\break\nv
\hfill
\faIcon{linkedin}\,\,\linkedin{kanggara75}\pipe
\faIcon{github}\,\,\github{kanggara75}
\hfill
\vspace{1em}
}

\newcommand{\resumeSection}[1]{
  \raggedright\MakeUppercase{\large\textbf{#1}}
  \break\vspace{-1em}
  \hrule
}

\newcommand{\resumeSectionItem}[4]{
  \normalsize\raggedright\textbf{#1}\hfill
  \small\raggedleft{#2}\\
  \small\raggedright{#3}\hfill
  \small\raggedleft{#4}\\
}

\begin{document}

% -------------------- HEADING--------------------
\begingroup
\resumeHeading{Kelvin Anggara}
\resumeInfo
\endgroup

% ABOUT
\begingroup
\resumeSection{TENTANG SAYA}
\normalsize\justifying{
	Saya memiliki pengalaman sebagai Backend Developer sejak 2022 dalam membangun dan mengelola sistem backend yang skalabel dan efisien.
	Saya juga memiliki pengalaman dalam Java Spring Boot, Docker, Kubernetes, dan event broker Solace.
	Saya memiliki keahlian dalam membangun RESTful API yang scalable, mengelola database MySQL/PostgreSQL dan CI/CD dengan GitHub Actions. 
	Saya telah memperluas pemahaman tentang konsep-konsep pemrograman dan pengembangan perangkat lunak melalui pengalaman kerja dan pembelajaran mandiri.
	Saya adalah seorang yang bersemangat dan terus berusaha untuk meningkatkan keterampilan saya dalam pengembangan perangkat lunak, serta siap untuk menghadapi tantangan baru dan berkontribusi dalam proyek-proyek yang inovatif.
}
\vspace{0.1em}
\endgroup

% Skills
\begingroup
\resumeSection{Keahlian}
\nv
\begin{multicols}{2}
  \normalsize\textbf{Soft Skills}\vspace{-1em}
  \small\justifying\begin{itemize}
    \item{Organisasi}\nv
    \item{Komunikasi}\nv
    \item{Berpikir Kritis}\nv
    \item{Manajemen Projek}\nv
    \item{Mudah Beradaptasi}\nv
    \item{Penyelesaian Masalah}\nv
    \item{Pengambilan Keputusan}\nv
    \item{Bekerja Dalam Tim dan Kolaborasi}\nv
  \end{itemize}
  \normalsize\textbf{Hard Skills}\vspace{-1em}
  \small\justifying\begin{itemize}
    \item{\textbf{Server   $:$} Linux \hfill Menengah}\nv
    \item{\textbf{Office   $:$} Word, Excel\hfill Mahir}\nv
    \item{\textbf{DevOps   $:$} Docker, K8s\hfill Menengah}\nv
    \item{\textbf{Mobile   $:$} Flutter (Dart)\hfill Menengah}\nv
    \item{\textbf{Database $:$} SQL, Redis\hfill Menengah}\nv
    \item{\textbf{CI/CD    $:$} Github Action\hfill Menengah}\nv
    \item{\textbf{Backend  $:$} Spring Boot (Java)\hfill Menengah}\nv
    % \item{\textbf{Networking $:$} Mikrotik\hfill Basic}\nv
    % \item{\textbf{IoT     $:$} ESP, Arduino\hfill Menengah}
  \end{itemize}
\end{multicols}

\endgroup

% Pengalaman Kerja
\begingroup
\resumeSection{Pengalaman Kerja}
\resumeSectionItem{XL Axiata, Tbk}{Jakarta Selatan, Indonesia}{\textbf{Backend Developer}}{Jul 2024 - Sekarang}
\nv\small\justifying\begin{itemize}
  \item{Menerjemahkan dokumen teknis/design ke dalam kode program sesuai dengan standard kode pemrogramman.}\nv
  \item{Merancang dan membuat user interface yang baik agar mudah dioperasikan.}\nv
  \item{Ikut terlibat dalam proses desain dan proses pengembangan aplikasi.}\nv
  \item{Melakukan unit test dan mendokumentasikan hasil dari unit test.}\nv
  \item{Terlibat dalam proses testing/integration test serta memastikan performance aplikasi berjalan dengan baik.}
\end{itemize}
\resumeSectionItem{SIGMA CIPTA CARAKA/TELKOMSIGMA}{Banten, Indonesia}{\textbf{Junior Programmer}}{Des 2022 - Jun 2024}
\nv\small\justifying\begin{itemize}
  \item{Menerjemahkan dokumen teknis/design ke dalam kode program sesuai dengan standard kode pemrogramman.}\nv
  \item{Merancang dan membuat user interface yang baik agar mudah dioperasikan.}\nv
  \item{Ikut terlibat dalam proses desain dan proses pengembangan aplikasi.}\nv
  \item{Melakukan unit test dan mendokumentasikan hasil dari unit test.}\nv
  \item{Terlibat dalam proses testing/integration test serta memastikan performance aplikasi berjalan dengan baik.}
\end{itemize}
\endgroup

% Project
\begingroup
\resumeSection{Projects}
\resumeSectionItem{MCCM Fulfillment}{\href{https://xl.co.id}{XL Axiata}}{\textbf{ }}{Jan 2025 - Feb 2025}
\vspace{0.5em}
\normalsize\justifying{
	Melakukan improvment pada sistem yang sudah ada dimana sebelumnya menggunakan metode sinkonronous menjadi asynchronous. Request yang di terima dari Even Broker akan di simpan ke dalam database terlebih dahulu sebelum di proses oleh worker yang sudah di sediakan.
}
\nv\small\justifying\begin{itemize}
  \item{Melakukan Tracing and Debugging selama proses development testing SIT, UAT hingga Go Live}
\end{itemize}
\resumeSectionItem{Comarch Redeem Reward}{\href{https://xl.co.id}{XL Axiata}}{\textbf{MyXL \& Axisnet}}{Jul 2024 - Des 2024}
\vspace{0.5em}
\normalsize\justifying{
	Melakukan pengembangan pada siste Loyalty yang memungkinkan pengguna untuk menukarkan poin yang telah di kumpulkan dari setiap transaksi yang dilakukan oleh pengguna.
	Fitur ini di implementasikan pada aplikasi MyXL dan Axisnet yang di kembangkan oleh tim XL Axiata.
}
\nv\small\justifying\begin{itemize}
  \item{Melakukan Tracing and Debugging selama proses development testing SIT, UAT hingga Go Live}
\end{itemize}
\resumeSectionItem{ATM Link HimBaRa JALIN eChannel Platform}{\href{https://telkomsigma.co.id}{TELKOMSIGMA}}{\textbf{ATMs and EFTs\small\textsuperscript{\textcopyright} Euronet Implementor}}{Des 2022 - Jun 2024}
\vspace{0.5em}
\normalsize\justifying{Bersama dengan Tim \href{https://telkomsigma.co.id}{Telkomsigma}, \href{https://www.jalin.co.id/}{Euronet Worldwide}, dan \href{https://www.jalin.co.id}{Jalin Pembayaran Nusantara} Mengembangkan Platform New ATM Link HimBaRa/Himpunan
  Bank Milik Negara (BNI, BRI, BTN, dan Bank Mandiri) dengan 335 Fitur yang menggunakan pendekatan berdasarkan Penerbit Kartu (Issuer Approach)
  yang di implementasikan pada 53 ribu Terminal ATM di seluruh Indonesia.
}
\nv\small\justifying\begin{itemize}
  \item{Implementasi ATMs Screen Flow berdasarkan FSD (Functional Specification Document)}\nv
  \item{Melakukan testing pada setiap fitur yang di kembangkan}\nv
  \item{Membuat dokumentasi dependency yang diperlukan untuk setiap fitur yang di kembangkan}\nv
  \item{Melakukan Tracing and Debugging selama proses development testing SIT, UAT hingga Go Live}
\end{itemize}
\endgroup

% Education
\begingroup
\resumeSection{PENDIDIKAN}
\resumeSectionItem{UIN SULTAN SYARIF KASIM RIAU}{Pekanbaru, Riau Indonesia}{{Sarjana Teknik – Teknik Elektro –} \textbf{IPK: 3.52/4.00}}{2016 - 2022}
\small\raggedright{Judul Tugas Akhir/Skripsi:}\\
\small\raggedright\textbf{Smart Early Warning System untuk keamanan Sepeda Motor Berbasis Prosesor XTensa LX6}\\
\vspace{0.5em}
\resumeSectionItem{SMK NUSANTARA}{Rokan Hilir, Riau Indonesia}{Teknik Komputer dan Jaringan}{2013 - 2016}
\break\nv
\endgroup

\end{document}
